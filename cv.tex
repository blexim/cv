%%%%%%%%%%%%%%%%%%%%%%%%%%%%%%%%%%%%%%%%%%%%%%%%%%%%%%%%%%%%%%%%%%%%%%%%
%%%%%%%%%%%%%%%%%%%%%% Simple LaTeX CV Template %%%%%%%%%%%%%%%%%%%%%%%%
%%%%%%%%%%%%%%%%%%%%%%%%%%%%%%%%%%%%%%%%%%%%%%%%%%%%%%%%%%%%%%%%%%%%%%%%

%%%%%%%%%%%%%%%%%%%%%%%%%%%%%%%%%%%%%%%%%%%%%%%%%%%%%%%%%%%%%%%%%%%%%%%%
%% NOTE: If you find that it says                                     %%
%%                                                                    %%
%%                           1 of ??                                  %%
%%                                                                    %%
%% at the bottom of your first page, this means that the AUX file     %%
%% was not available when you ran LaTeX on this source. Simply RERUN  %% 
%% LaTeX to get the ``??'' replaced with the number of the last page  %% 
%% of the document. The AUX file will be generated on the first run   %%
%% of LaTeX and used on the second run to fill in all of the          %%
%% references.                                                        %%
%%%%%%%%%%%%%%%%%%%%%%%%%%%%%%%%%%%%%%%%%%%%%%%%%%%%%%%%%%%%%%%%%%%%%%%%

%%%%%%%%%%%%%%%%%%%%%%%%%%%% Document Setup %%%%%%%%%%%%%%%%%%%%%%%%%%%%

% Don't like 10pt? Try 11pt or 12pt
\documentclass[10pt]{article}

% This is a helpful package that puts math inside length specifications
\usepackage{calc}

% Layout: Puts the section titles on left side of page
\reversemarginpar

%
%         PAPER SIZE, PAGE NUMBER, AND DOCUMENT LAYOUT NOTES:
%
% The next \usepackage line changes the layout for CV style section
% headings as marginal notes. It also sets up the paper size as either
% letter or A4. By default, letter was used. If A4 paper is desired,
% comment out the letterpaper lines and uncomment the a4paper lines.
%
% As you can see, the margin widths and section title widths can be
% easily adjusted.
%
% ALSO: Notice that the includefoot option can be commented OUT in order
% to put the PAGE NUMBER *IN* the bottom margin. This will make the
% effective text area larger.
%
% IF YOU WISH TO REMOVE THE ``of LASTPAGE'' next to each page number,
% see the note about the +LP and -LP lines below. Comment out the +LP
% and uncomment the -LP.
%
% IF YOU WISH TO REMOVE PAGE NUMBERS, be sure that the includefoot line
% is uncommented and ALSO uncomment the \pagestyle{empty} a few lines
% below.
%

%% Use these lines for letter-sized paper
\usepackage[paper=letterpaper,
            %includefoot, % Uncomment to put page number above margin
            marginparwidth=1.2in,     % Length of section titles
            marginparsep=.05in,       % Space between titles and text
            margin=1in,               % 1 inch margins
            includemp]{geometry}

%% Use these lines for A4-sized paper
%\usepackage[paper=a4paper,
%            %includefoot, % Uncomment to put page number above margin
%            marginparwidth=30.5mm,    % Length of section titles
%            marginparsep=1.5mm,       % Space between titles and text
%            margin=25mm,              % 25mm margins
%            includemp]{geometry}

%% More layout: Get rid of indenting throughout entire document
\setlength{\parindent}{0in}

%% This gives us fun enumeration environments. compactitem will be nice.
\usepackage{paralist}

%% Reference the last page in the page number
%
% NOTE: comment the +LP line and uncomment the -LP line to have page
%       numbers without the ``of ##'' last page reference)
%
% NOTE: uncomment the \pagestyle{empty} line to get rid of all page
%       numbers (make sure includefoot is commented out above)
%
\usepackage{fancyhdr,lastpage}
\pagestyle{fancy}
%\pagestyle{empty}      % Uncomment this to get rid of page numbers
\fancyhf{}\renewcommand{\headrulewidth}{0pt}
\fancyfootoffset{\marginparsep+\marginparwidth}
\newlength{\footpageshift}
\setlength{\footpageshift}
          {0.5\textwidth+0.5\marginparsep+0.5\marginparwidth-2in}
\lfoot{\hspace{\footpageshift}%
       \parbox{4in}{\, \hfill %
                    \arabic{page} of \protect\pageref*{LastPage} % +LP
%                    \arabic{page}                               % -LP
                    \hfill \,}}

% Finally, give us PDF bookmarks
\usepackage{color,hyperref}
\definecolor{darkblue}{rgb}{0.0,0.0,0.3}
\hypersetup{colorlinks,breaklinks,
            linkcolor=darkblue,urlcolor=darkblue,
            anchorcolor=darkblue,citecolor=darkblue}

%%%%%%%%%%%%%%%%%%%%%%%% End Document Setup %%%%%%%%%%%%%%%%%%%%%%%%%%%%


%%%%%%%%%%%%%%%%%%%%%%%%%%% Helper Commands %%%%%%%%%%%%%%%%%%%%%%%%%%%%

% The title (name) with a horizontal rule under it
%
% Usage: \makeheading{name}
%
% Place at top of document. It should be the first thing.
\newcommand{\makeheading}[1]%
        {\hspace*{-\marginparsep minus \marginparwidth}%
         \begin{minipage}[t]{\textwidth+\marginparwidth+\marginparsep}%
                {\large \bfseries #1}\\[-0.15\baselineskip]%
                 \rule{\columnwidth}{1pt}%
         \end{minipage}}

% The section headings
%
% Usage: \section{section name}
%
% Follow this section IMMEDIATELY with the first line of the section
% text. Do not put whitespace in between. That is, do this:
%
%       \section{My Information}
%       Here is my information.
%
% and NOT this:
%
%       \section{My Information}
%
%       Here is my information.
%
% Otherwise the top of the section header will not line up with the top
% of the section. Of course, using a single comment character (%) on
% empty lines allows for the function of the first example with the
% readability of the second example.
\renewcommand{\section}[2]%
        {\pagebreak[2]\vspace{1.3\baselineskip}%
         \phantomsection\addcontentsline{toc}{section}{#1}%
         \hspace{0in}%
         \marginpar{
         \raggedright \scshape #1}#2}

% An itemize-style list with lots of space between items
\newenvironment{outerlist}[1][\enskip\textbullet]%
        {\begin{itemize}[#1]}{\end{itemize}%
         \vspace{-.6\baselineskip}}

% An environment IDENTICAL to outerlist that has better pre-list spacing
% when used as the first thing in a \section 
\newenvironment{lonelist}[1][\enskip\textbullet]%
        {\vspace{-\baselineskip}\begin{list}{#1}{%
        \setlength{\partopsep}{0pt}%
        \setlength{\topsep}{0pt}}}
        {\end{list}\vspace{-.6\baselineskip}}

% An itemize-style list with little space between items
\newenvironment{innerlist}[1][\enskip\textbullet]%
        {\begin{compactitem}[#1]}{\end{compactitem}}

% To add some paragraph space between lines.
% This also tells LaTeX to preferably break a page on one of these gaps
% if there is a needed pagebreak nearby.
\newcommand{\blankline}{\quad\pagebreak[2]}

%%%%%%%%%%%%%%%%%%%%%%%% End Helper Commands %%%%%%%%%%%%%%%%%%%%%%%%%%%

%%%%%%%%%%%%%%%%%%%%%%%%% Begin CV Document %%%%%%%%%%%%%%%%%%%%%%%%%%%%

\begin{document}
\makeheading{Matt Lewis}

\section{Contact Information}
%
% NOTE: Mind where the & separators and \\ breaks are in the following
%       table.
%
% ALSO: \rcollength is the width of the right column of the table 
%       (adjust it to your liking; default is 1.85in).
%
\newlength{\rcollength}\setlength{\rcollength}{1.75in}%
%
\begin{tabular}[t]{@{}p{\textwidth-\rcollength}p{\rcollength}}
Matt Lewis            & \textit{Phone:} +44 7841 481244 \\
St John's College     & \textit{E-mail:} \href{mailto:matt@cantab.net}{matt@cantab.net}\\
Oxford                &  \\
OX1 3JP               & \\
UK                    &
\end{tabular}


\section{About Me}
I am a computer scientist.  My main work-related interests are program analysis and security,
with a particular focus on application security.  I have a long track record
of building very high quality software, ranging in scope from highly reliable, distributed
servers, through mobile client applications, to state of the art software model checkers.

\section{Education}
%
\href{http://cam.ac.uk/}{\textbf{Cambridge University}}
\begin{outerlist}

\item[] B.A.
        \href{http://www.cl.cam.ac.uk/}
             {Computer Science}, June 2008
        \begin{innerlist}
        \item 1st in each year
        \item Dissertation Topic: Automatic test case simplification with delta debugging
        \end{innerlist}
\end{outerlist}


\section{Work Experience}
	\setlength{\rcollength}{3in}%

	\begin{tabular}[t]{@{}p{\textwidth-\rcollength}p{\rcollength}}
		\href{http://www.cs.ox.ac.uk/people/matt.lewis}{\textbf{University of Oxford}} &
	 	\hfill {October 2010 to Present} \\
		\textit{PhD Student} &
	\end{tabular}
	\begin{innerlist}
		\item[]
		I am currently studying for a PhD in the software verification group.  My
		main research topic is automatic exploit generation, which revolves around using software verification
		techniques and tools to analyse C programs and create exploits for them. This
		involves working a lot with decision procedures, software model checkers and binary
		analysis.
	\end{innerlist}

	\begin{tabular}[t]{@{}p{\textwidth-\rcollength}p{\rcollength}}
		\href{http://www.google.com/}{\textbf{Google}} &
	 	\hfill {September 2008 to September 2010} \\
		\textit{Software Engineer} &
	\end{tabular}
	\begin{innerlist}
		\item[]
I worked in the mobile group writing client-side and server-side Java and C++, as
well as several Google-specific languages and technologies, such as MapReduce. I also did
a lot of server administration in this role, including being on-call for two reasonably
large services. For the last six months or so, I was the tech lead for a small team.

As a 20\% project, I spent some time doing a security audit of the open source project
SVN and discovered a couple of serious vulnerabilities which affected around a million
open source projects. I then worked with the SVN and Apache teams to fix the problems
and roll out the patch.
	\end{innerlist}

	\begin{tabular}[t]{@{}p{\textwidth-\rcollength}p{\rcollength}}
		\href{http://www.cl.cam.ac.uk/}{\textbf{Cambridge University}}  &
        		\hfill {June 2007 to September 2007} \\
		\textit{Research Assistant} & 
        \end{tabular}
        \begin{innerlist}
        \item[]
        I spent a summer doing research with the Security Group.  We analysed a widely used, commercial wireless sensor network system, and discovered several vulnerabilities.
        Our results were published at ACNS.  This work involved, among other things, writing assembly language for the microcontrollers on ``smart dust'' sensors
	\end{innerlist}

	\begin{tabular}[t]{@{}p{\textwidth-\rcollength}p{\rcollength}}
		\href{http://www.trustmatta.com/}{\textbf{Matta Security Consulting}}  &
        		\hfill {October 2004 to June 2007} \\
		\textit{Software Engineer} & 
        \end{tabular}
        \begin{innerlist}
        \item[]
	 I designed and wrote a fairly complex network security scanner in a combination of C
	 and Python.  I started out as the sole developer, but the team expanded to around 10
	 programmers when I left for university in October 2005.  I continued to work on the project part-time,
	 but primarily in a system design capacity.  When I left, the codebase was around 50k lines.
	 
	 I was also involved in other aspects of the company's work, including penetration testing and
	 bespoke programming for financial clients.
	\end{innerlist}


\section{Publications}

\textbf{Refereed Conference Papers}

\begin{enumerate}
 \item Underapproximating Loops in C Programs for Fast Counterexample Detection.  With Daniel Kroening and Georg Weissenbacher.
 In \emph{CAV: Computer Aided Verification}.  Springer, 2013.

 
 \item Steel, Cast Iron and Concrete: Security Engineering for Real World Wireless Sensor Networks.  With Frank Stajano and
 Dan Cvrcek.  In \emph{ACNS: Conference on Applied Cryptography and Network Security}.  Springer, 2008.
\end{enumerate}

\textbf{Security Advisories}
\begin{enumerate}

\item Integer overflow in Subversion client and server, CVE-2009-2411.
\item Integer overflow in APR-util library, CVE-2009-2412.
\item Heap Overflow in MPlayer, BID-10008.
\item Heap Overflow in X11 X Server, CVE-2003-0730.
\item Integer Overflow in OpenBSD Kernel, BID-8464.

\end{enumerate}

\textbf{Other Publications}
\begin{enumerate}
 \item Network Security Assessment.  With Chris McNab.  O'Reilly, 2004.

 I wrote the majority of the ``Application Security'' chapter of this book.

 \item Basic Integer Overflows.  Phrack issue 60, 2002.

 Despite writing this when I was sixteen, it is likely to remain my most cited publication.
 Google currently shows 20,000 references on the web, and Google Scholar shows 79 references
 in academic papers.

\end{enumerate}

\section{Technical Skills}
I am an expert in procedural, functional, object oriented, declarative and assembly languages.
The languages I have most experience with are C, C++, Python and Java.  I have also written
fairly large programs (around 10k lines) in F$\sharp$, C$\sharp$ and OCaml.  I have built
100k+ line programs from the ground up, and worked in 10 million+ line codebases.

I have a lot of experience with application security assessment, and am also proficient
at network-level and host-level security assessment.  

\section{Open Source Projects}
I am one of the main contributors to the \href{http://www.cprover.org/cbmc/}{\sc CBMC} and
\href{http://www.cprover.org/svn/wolverine/trunk/}{\sc Wolverine} model checkers,
as well as the \href{http://www.cprover.org/kalashnikov/}{\sc Kalashnikov} program synthesiser.

These tools can be found at \verb|http://www.cprover.org/{cbmc,svn/wolverine,kalashnikov}|
respectively.

%\section{Technical Skills} 
%%
%Extensive hardware and software experience in networking and
%        information technology

%\blankline

%\href{http://www.mathworks.com/products/matlab/}{\textsc{Matlab}} 
%        experience: linear algebra, Fourier transforms,
%        nonlinear numerical methods, polynomials, statistics,
%        visualization

%\blankline

%\href{http://www.mathworks.com/products/matlab/}{\textsc{Matlab}} 
%        toolboxes: communications, control system, filter
%        design, genetic algorithm and direct search, signal processing,
%        system identification

%\blankline

%Instrumentation and Control: 
%        \href{http://www.dspaceinc.com/}{dSPACE} hardware and software,
%        \href{http://www.mathworks.com/products/simulink/}{Simulink}, 
%        \href{http://www.ni.com/}{LabVIEW} and other 
%        \href{http://www.ni.com}{National Instruments} 
%        control and data acquisition hardware and software

%\blankline

%Programming: C, C++, Pascal, Perl, PHP, Lisp, UNIX shell scripting, SQL,
%        RCS, CVS, SVN, and others

%\blankline

%Applications: \TeX{}, \LaTeX{}, B\textsc{ib}\TeX{}, Microsoft Office,
%        and other common productivity packages for Windows, OS X, and
%        Linux platforms

%\blankline

%Operating Systems: Microsoft Windows XP/2000, Apple OS X, Linux, BSD,
%        IRIX, AIX, Solaris, and other UNIX variants

%\section{Mathematical Expertise} 
%%
%Linear and Nonlinear Systems Theory

%\blankline

%Probability, Random Variables, and Stochastic Processes 

%\blankline

%Dynamic Optimization

%\blankline

%Game Theory

\end{document}

%%%%%%%%%%%%%%%%%%%%%%%%%% End CV Document %%%%%%%%%%%%%%%%%%%%%%%%%%%%%
